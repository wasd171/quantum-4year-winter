\documentclass[12pt]{customArticle}

\usepackage[explicit]{titlesec}
\titleformat{\section}
  {\normalsize}{\bf \thesection.}{1em}{\bf #1}
\usepackage{braket}
\usepackage{framed}


\title{Ответы к вопросам по курсу \\ ``Квантовая теория''}

\begin{document}
	\maketitle
		
	\section{Соотношения неопределенностей (для операторов координаты и импульса, для компонент вектора момента). Когерентные состояния гармонического осциллятора. (L5, L17, L38)}
	
		\begin{enumerate}[label=\asbuk*)]
			\item {
				Рассмотрим дисперсию величины $A: \quad \overline{\Delta A^2} = \Bra{\psi} \left(\hat{A} - \bar{A} \right)^2 \Ket{\psi}$ 
				\\
				Дисперсия значений величин $x$ и $y$ связана с коммутатором операторов: $\left[ \hat{x}, \hat{y} \right] = i \hat{A}$
				\\
				Пусть оператор $\hat{L} = \left( \hat{x} - \bar{x} \right) + i \gamma \left( \hat{y} - \bar{y} \right), \quad \gamma \in {\Bbb R}, \quad ! \hat{x}, \hat{y} \text{ -- эрмитовы} $
				\\
				Собственные значения оператора $\hat{L}^+ \hat{L}$ неотрицательны:
				\begin{gather*}
					\Bra{\psi} \left\{ \left( \hat{x} - \bar{x} \right) - i \gamma \left( \hat{y} - \bar{y} \right) \right\} \left\{ \left( \hat{x} - \bar{x} \right) + i \gamma \left( \hat{y} - \bar{y} \right) \right\} \Ket{\psi} \geq 0
					\\
					\Bra{\psi} \overline{\Delta x^2} + \gamma^2 \overline{\Delta y^2} - \gamma \bar{A} \Ket{\psi} \geq 0
				\end{gather*}
				Выполняется для $\forall \gamma \Rightarrow D \leq 0 : \quad \bar{A}^2 - 4 \overline{\Delta x^2} \; \overline{\Delta y^2} \leq 0$
				\begin{center}
					\boxed{
						\overline{\Delta x^2} \; \overline{\Delta y^2} \geq \frac{\bar{A}^2}{4}
					}
				\end{center}
				
				Таким образом $\displaystyle \left[ \hat{p_i}, \hat{x_i} \right] = -i \hbar \quad \Rightarrow \quad \underline{ \overline{\Delta p^2} \; \overline{\Delta x^2} \geq \frac{\hbar^2}{4} }$
				\\
				Аналогично для векторного оператора момента: 
				\begin{gather*}
					\left[ \hat J_i , \hat J_k \right] = i \Epsilon_{ijk} \hat J_j \Rightarrow \Delta J_x \Delta J_y \geq \frac{1}{2} \left| \left< \hat J_z \right> \right|
					\\
					\hat J_x = \frac{1}{2} \begin{vmatrix} 0 & 1 & 0 \\ 1 & 0 & 1 \\ 0 & 1 & 0\end{vmatrix}; \quad \hat J_y = \frac{-i}{2} \begin{vmatrix}
						0 & 1 & 0 \\
						-1 & 0 & 1 \\
						0& -1 & 0
					\end{vmatrix}; \quad \hat J_z = \begin{vmatrix}
						1 & 0 & 0 \\
						0 & 0 & 0 \\
						0 & 0 & -1
					\end{vmatrix}
					\\
					\Epsilon_{ijk} \text{ -- единичный антисимметричный тензор (Леви-Чивиты)}
				\end{gather*}
			}
			\item {
				В классической механике $p$ и $x$ определены одновременно $\Rightarrow$ состояния, наиболее близкие к классическим описываются минимальными неопределённостями, то есть \[ D = 0 \quad \text{и} \quad \overline{\Delta p^2} \; \overline{\Delta x^2} = \frac{\hbar^2}{4} \]
				Так как СЗ оператора $\hat L^+ \hat L$ неотрицательны, то СФ оператора $L^+$ будут соответствовать СЗ $= 0$. Получаем:
				\[
					\brround{x + \gamma \hbar \frac{\partial}{\partial x}} \psi = \lambda \psi, \quad \lambda = \bar x + i \gamma \bar p
				\]
				Решение этого уравнения: $\psi(x) = \brround{2 \pi \overline{\Delta x^2} }^{-\frac{1}{4}} \exp \brsquare{- \frac{ \brround{x - \bar x }^2 }{4 \overline{ \Delta x^2 } } + i \frac{\bar p x}{\hbar}  }$
				\\
				Если забить на размерности, то это уравнение эквивалентно 
				\begin{gather*}
					\hat a \psi = \lambda \psi; \quad \hat a \Ket{\alpha} = \alpha \Ket{\alpha}; \quad \Braket{\alpha|\alpha} = 1
					\\
					\hat a = \frac{1}{\sqrt{2}} \brround{\hat x + i \hat p} \text{ -- оператор уничтожения}
				\end{gather*}
				Последнему уравнению удовлетворяют ВФ, называющиеся {\bf когерентными состояниями гармонического осциллятора}
			}
		\end{enumerate}
		
	\section{Динамика квантовых систем в картине Гейзенберга. Теоремы Эренфеста. Эволюция волновых пакетов свободной частицы. (L7)}
		\begin{enumerate}[label=\asbuk*)]
			\item {
				Матричные элементы операторов $\hat x_i$ и $\hat p_k$, вычисленные между ВФ $f$ и $g$ удовлетворяют уравнениям Гамильтона классической механики:
		\begin{gather*}
			\frac{d}{dt} \Braket{f \left| \hat p_i \right| g} = - \Braket{f \left| \frac{\partial \hat H}{\partial \hat x_i} \right| g}
			\\
			\frac{d}{dt} \Braket{f \left| \hat x_i \right| g} = - \Braket{f \left| \frac{\partial \hat H}{\partial \hat p_i} \right| g}
		\end{gather*}
		{\bf Картина Гейзенберга:} от времени зависят лишь операторы, то есть
		\[
			\dot{\widehat{p_i}} = - \frac{\partial \hat H}{\partial \hat x_i} 
			\qquad
			\dot{\widehat{x_i}} = - \frac{\partial \hat H}{\partial \hat p_i} 
		\]
		Учитывая, что $\displaystyle \hat p_i \hat{x_i}^n - \hat{x_i}^n \hat p_i = -i \hbar n \hat{x_i}^{n-1}$ для всех функций, разлагая в степенной ряд, получаем
		\[
			\hat p_i f - f \hat p_i = -i \hbar \frac{\partial f}{\partial x_i} \Rightarrow 
			\begin{cases}
				\dot{\widehat p_i} = - \frac{i}{\hbar} \brsquare{\widehat p_i, \widehat H}
				\\[1em]
				\dot{\widehat x_i} = - \frac{i}{\hbar} \brsquare{\widehat x_i, \widehat H}

			\end{cases}
			\leftarrow \quad \textbf{уравнения Гейзенберга}
		\]
	
			}
			
			\item {
				\begin{enumerate}[label=\arabic*)]
					\item{
						\[
							\frac{d}{dt} \left< x \right> = \frac{i}{\hbar} \left< Hx - xH\right>, \qquad H = \frac{p^2_x}{2m} + V(x)
						\]
						Так как $x$ и $V(x)$ коммутируют, то $\frac{d}{dt} \left< x \right> = \frac{i}{2m \hbar} \left< p^2_x - x p^2_x\right> $
						\[
							\frac{d}{dt} \left< x \right> = \frac{i}{2m \hbar} \left< p_x \left(p_x x - x p_x\right) + \left(p_x x - x p_x\right) p_x \right>
						\]
						Далее так как $p_x x - x p_x = -i \hbar$, то \boxed{\frac{d}{dt} \left< x \right> = \frac{1}{m} \left< p_x \right>}
					}
					\item{
						\[
							\frac{d}{dt} \left< p_x \right> = \frac{i}{\hbar} \left< V p_x - p_x V \right> = - \Braket{\frac{\partial V}{\partial x}}
						\]
						\begin{center}
							\boxed{
							\frac{d}{dt} \left< p_x \right> = m \frac{d^2}{d t^2} \left< x \right> = -\Braket{\frac{\partial V}{\partial x}} = \Braket{F(x)}
						}
						\end{center}
					}
				\end{enumerate}
				Формулы в коробочках -- теорема Эренфеста
			}
			\item{
				Свободная частица $\displaystyle \widehat H = \frac{\widehat p^2}{2m}$. Решение: 
				\[
					\widehat p (t) = \widehat p_0; \qquad \widehat r (t) = \widehat r_0 + \frac{1}{m} \widehat p_0 t
				\]
			}
		\end{enumerate}
		
	\section{Динамика квантовых систем в картине Шредингера. Эволюция фиделити. Теорема Крылова – Фока. (L8)}
	
		\begin{enumerate}[label=\asbuk*)]
			\item{
				Пусть теперь от времени зависят ВФ $f, g$:
				\begin{gather*}
					\frac{d}{dt} \Braket{f \left| \widehat p_i \right| g} = - \frac{i}{\hbar} \Braket{f \left|  \brsquare{\widehat p_i, \widehat H} \right| g}
					\\
					\brround{ \fracpart{f}{t} , \widehat p_i g } + \brround{ \widehat p_i f, \fracpart{g}{t} } = \frac{i}{\hbar} \brround{f, \widehat p_i \widehat H g} + \frac{i}{\hbar} \brround{f, \widehat H \widehat p_i g}
					\\
					\widehat H, \widehat p \text{ -- эрмитовы} \quad \Rightarrow
					\\
					\brround{ \fracpart{f}{t} , \widehat p_i g } + \brround{ \widehat p_i f, \fracpart{g}{t} } = -\frac{i}{\hbar} \brround{\widehat p_i f, \widehat H g} + \frac{i}{\hbar} \brround{\widehat H f, \widehat p_i g}
					\\
					\brround{\fracpart{f}{t} + \frac{i}{\hbar} \widehat H f, \widehat p_i g } + \brround{\widehat p_i f, \fracpart{g}{t} + \frac{i}{\hbar} \widehat H g} = 0
				\end{gather*}
				Выполняется для ВФ, удовлетворяющих \boxed{i \hbar \fracpart{\psi}{t} = \widehat H \psi} $\leftarrow$ УШ	
			}
			\item{
				Суперпозиция двух стационарных состояний: 
				\[
					\psi (t) = \alpha \Ket{\psi_k} \exp \brround{- i \omega_k t} + \beta \Ket{\psi_n} \exp \brround{-i \omega_n t}
				\]
				Фиделити текущего и начального состояний:
				\begin{gather*}
					\mathcal{F}(t) = \left| \Braket{\psi(t) | \psi(0)} \right|^2 = |\alpha|^4 + 2|\alpha|^2 |\beta|^2 \cos \omega_{nk} t + |\beta|^4
					\\
					\omega_{nk} = \frac{E_n - E_k}{\hbar} = \omega_n - \omega_k \quad \text{-- ``частота перехода''}
				\end{gather*}
				
				Для непрерывного спектра $E$ суперпозиция состояний:
				\[
					\psi(\xi, 0) = \int C(E) \psi_E (\xi) dE
				\]
				Фиделити текущего и начального состояний:
				\[
					\boxed{
						\mathcal{F}(t) = \left|  \Braket{\psi(\xi, 0) | \psi(\xi, t)} \right|^2 = \left| \int \left| C(E) \right|^2 \exp \brround{-\frac{i}{\hbar} Et} dE \right|^2
					}
					\leftarrow
					\text{теорема Крылова-Фока}
				\]
			}
		\end{enumerate}
		
	\section{Матричные элементы операторов координаты и импульса. Их связь, теоремы о суммах. Теорема соответствия для матричных элементов. (L8, L9, L15)}
	
		\begin{enumerate}[label=\asbuk*)]
			\item{
				Общий вид уравнения Гейзенберга: $\dot{\widehat z} = \frac{i}{\hbar} \brsquare{\widehat H, \widehat z}$
				\[
					\Rightarrow \boxed{ \brround{\dot Z}_{nk} = i \omega_{nk} z_{nk} }
					\qquad
					p_{nk} = i m \omega_{nk} x_{nl}
				\]
			}
			\item{
				Теорема о матричных элементах: из а) следует, что
				\begin{gather*}
					A_m = \sum_k \brlines{z_{nk}}^2 \omega^{2m+1}_{kn} = -\frac{i}{2} \Braket{n \brlines{ \brsquare{\frac{d^m \widehat z}{d t^m}, \frac{d^{m+1} \widehat z}{d t^{m+1}}} } n}
					\\
					B_m = \sum_k \brlines{z_{kn}}^2 \omega^{2m}_{kn} = \Braket{n  \brlines{ \brround{\frac{d^m \widehat z}{d t^m}}^2 } n}
				\end{gather*}
				Ещё можно почитать в L9
			}
			\item{
				\textbf{Теорема о соответствии матричных элементов:}
				\\
				Матричный элемент координаты $x_{n+p, n}$ между стационарными состояниями $\approx$ фурье-амплитуде $X_p$ $p$-й гармоники закона классического движения при энергии, равной среднему значению от энергий начального и конечного состояний перехода -- с точностью до $\hbar$
				\[
					x_{n+p, n} \approx X_p + \hbar \frac{p}{2} \Omega \frac{d X_p}{dE}, \qquad \Omega \text{ -- частота движения классической системы}
				\]
			}
		\end{enumerate}
	
				
		
		
		
		
		
		
		
		
		
		
		
		
		
		
		
		
		
		
		
		
		
		
		
		
		
		
		
		
		
		
		
		
		
		
		
		
		
		
		
\end{document}